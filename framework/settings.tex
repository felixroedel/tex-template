% Set default graphics path
\graphicspath{{img/}}

%% Disable single lines at the start of a paragraph (Schusterjungen)
\clubpenalty=10000

% Disable single lines at the end of a paragraph (Hurenkinder)
\widowpenalty=10000
\displaywidowpenalty=10000

% Fixed size table columns
\newcolumntype{L}[1]{>{\raggedright\arraybackslash}p{#1}}
\newcolumntype{C}[1]{>{\centering\arraybackslash}p{#1}}
\newcolumntype{R}[1]{>{\raggedleft\arraybackslash}p{#1}}

\renewcommand{\arraystretch}{1.2}                       % Distance between lines in tables

% \captionsetup{justification=raggedright,singlelinecheck=false}              % Align captions to the left

\setcounter{secnumdepth}{3}                             % Only allow nesting 3 layers (down to subsubsections)

% Unit-related setup for siunitx
\sisetup{
  locale=DE,                   % Use German notation (comma instead of point delimiter for floats)
  per-mode=fraction,           % Switch display to use \frac instead of x^{-1}
  fraction-function=\tfrac,    % Use amsmath's tfrac macro for unit fractions
}

% -------------------------------------------------------------------
%                     Usability & visual changes
% -------------------------------------------------------------------

% Page margins
\geometry{
  left=2.5cm,
  right=2.5cm,
  top=2.5cm,
  bottom=3cm
}

% Create a better looking header and footer
\pagestyle{fancy}
\fancyhf{}
\lhead{\nouppercase{\leftmark}}
\fancyfoot[c]{\thepage}
\newcommand{\lheader}[1]{\fancyhead[L]{\raisebox{-0.1\height}{\includegraphics[width=4cm]{#1}}}}
\newcommand{\rheader}[1]{\fancyhead[R]{#1}}

% Automatically generate a box around figure environments
% \floatstyle{boxed}
% \restylefloat{figure}

% Set toc sections to be clickable
\hypersetup{
  colorlinks,
  citecolor=black,
  filecolor=black,
  linkcolor=black
}

\frenchspacing                                                 % Insert one space after a sentence, not 2

\renewcommand{\theequation}{\thesection.\arabic{equation}}     % Numbering of equation schould be x.x
\renewcommand{\UrlFont}{\color{blue}\rmfamily\itshape}         % URLs should be displayed in blue
\renewcommand{\dateseparator}{.}                               % Dates are written like 01.01.1970, not 01-01-1970


% -------------------------------------------------------------------
%                        Code listing setup
% -------------------------------------------------------------------

\lstset{
  basicstyle=\small\ttfamily\color{black},              % Font size used for the code
  commentstyle=\ttfamily\color{gray},                   % Comment style
  keywordstyle=\ttfamily\color{blue},                   % Keyword style
  stringstyle=\color{ForestGreen!30!LimeGreen},         % String literal style
  frame=single,                                         % Add a frame around the code
  showstringspaces=false,                               % Don't underline spaces within strings only
  captionpos=b,                                         % Set caption-position to bottom
  backgroundcolor=\color{white},                        % Background color
}

% To style lstlistlisting like the lof, you first have to register it
% to tocloft, as mentioned in https://tex.stackexchange.com/a/27648/27635
\makeatletter
\begingroup\let\newcounter\@gobble\let\setcounter\@gobbletwo
  \globaldefs\@ne \let\c@loldepth\@ne
  \newlistof{listings}{lol}{\lstlistlistingname}
\endgroup
\let\l@lstlisting\l@listings
\makeatother

% -------------------------------------------------------------------
%                         Redefining geometry
% -------------------------------------------------------------------

% Figures
\renewcommand{\cftfigpresnum}{\figurename~}
\renewcommand{\cftfigaftersnum}{:}
\setlength{\cftfignumwidth}{2cm}
\setlength{\cftfigindent}{0cm}

% Tables
\renewcommand{\cfttabpresnum}{\tablename~}
\renewcommand{\cfttabaftersnum}{:}
\setlength{\cfttabnumwidth}{2cm}
\setlength{\cfttabindent}{0cm}

% Listings
\renewcommand*{\cftlistingspresnum}{\lstlistingname~}
\renewcommand*{\cftlistingsaftersnum}{:}
\settowidth{\cftlistingsnumwidth}{\cftlistingspresnum}
\addtolength{\cftlistingsnumwidth}{1cm}
\setlength{\cftlistingsindent}{0cm}

\setlength{\parindent}{0cm}                             % Don't indent start of paragraph
\setlength{\parskip}{6pt}                               % Lines are separated by 6pt

\setlength{\headheight}{1.25cm}
\setlength{\footskip}{1cm}
\setlength{\headsep}{1cm}

% -------------------------------------------------------------------
%                    Custom counters & commands
% -------------------------------------------------------------------

% Custom acronym patches:
%   - count usage (used for on-demand list of acronyms)
%   - provide shorter aliases for commonly used macros
\newcounter{countacronym}
\DeclareTotalCounter{countacronym}
\pretocmd{\acrshort}{\stepcounter{countacronym}}{}{}
\pretocmd{\acrlong}{\stepcounter{countacronym}}{}{}
\pretocmd{\acrfull}{\stepcounter{countacronym}}{}{}
\newcommand*{\acr}[1]{\acrshort{#1}}
\newcommand*{\Acr}[1]{\acrlong{#1}}

% Custom nomenclature patches:
%   - count usage of actual entries through \nomen (used for on-demand nomenclature)
%   - provide cleaner writing interface with pre-made abstractions
\newcounter{countnomen}
\DeclareTotalCounter{countnomen}
\newcommand*{\nomen}[2]{\nomenclature{#1}{#2}\stepcounter{countnomen}}
\newcommand{\nomunit}[1]{\renewcommand{\nomentryend}{\hspace*{\fill}#1}}
\newcommand{\nomsi}[1]{\nomunit{[\si{#1}]}}

% Custom glossary patches:
%   - count usage (used for on-demand glossary)
\newcounter{countglossary}
\DeclareTotalCounter{countglossary}
\pretocmd{\Gls}{\stepcounter{countglossary}}{}{}
\pretocmd{\gls}{\stepcounter{countglossary}}{}{}
\pretocmd{\Glspl}{\stepcounter{countglossary}}{}{}
\pretocmd{\glspl}{\stepcounter{countglossary}}{}{}

% Custom citation patches:
%   - count usage (used for on-demand bibliography)
\newcounter{countcitation}
\DeclareTotalCounter{countcitation}
\pretocmd{\cite}{\stepcounter{countcitation}}{}{}

% TODO annotations
\newcounter{counttodo}
\DeclareTotalCounter{counttodo}
\pretocmd{\todo}{\stepcounter{counttodo}}{}{}
\newcommand{\note}   [2][]{\todo[color=green!25,bordercolor=green,tickmarkheight=3pt,#1]{#2}}
\newcommand{\unsure} [2][]{\todo[color=Plum!25,bordercolor=Plum,tickmarkheight=3pt,#1]{#2}}
\newcommand{\change} [2][]{\todo[color=blue!25,bordercolor=blue,tickmarkheight=3pt,#1]{#2}}
\newcommand{\missing}[2][]{\todo[color=red!25,bordercolor=red,tickmarkheight=3pt,#1]{#2}}
\newcommand{\info}   [2][]{\todo[nolist,color=yellow!25,bordercolor=yellow,tickmarkheight=3pt,#1]{#2}}

% Larger clickable references to page items
\newcommand{\imgref} [1]{\hyperref[#1]{Abb.~\getrefnumber{#1}}}
\newcommand{\tabref} [1]{\hyperref[#1]{Tab.~\getrefnumber{#1}}}
\newcommand{\coderef}[1]{\hyperref[#1]{Code~\getrefnumber{#1}}}
\newcommand{\mathref}[1]{\hyperref[#1]{Gl.~\getrefnumber{#1}}}

\newcommand{\Imgref} [1]{\hyperref[#1]{Figure~\getrefnumber{#1}}}
\newcommand{\Tabref} [1]{\hyperref[#1]{Table~\getrefnumber{#1}}}
\newcommand{\Coderef}[1]{\hyperref[#1]{Code~\getrefnumber{#1}}}
\newcommand{\Mathref}[1]{\hyperref[#1]{Equation~\getrefnumber{#1}}}
\newcommand{\secref} [1]{\autoref{#1}}

% -------------------------------------------------------------------
%                    Configure Tikz library
% -------------------------------------------------------------------
\usetikzlibrary{shapes.geometric, arrows}
\tikzstyle{startstop} = [rectangle,
                         rounded corners,
                         minimum width=3cm,
                         minimum height=1cm,
                         text centered,
                         draw=black,
                         fill=red!30
                        ]
\tikzstyle{io} = [trapezium,
                  trapezium left angle=70,
                  trapezium right angle=110,
                  minimum width=3cm,
                  minimum height=1cm,
                  text centered,
                  draw=black,
                  fill=blue!30
                ]
\tikzstyle{process} = [rectangle,
                       minimum width=3cm,
                       minimum height=1cm,
                       text width=8cm,
                       text centered,
                       draw=black,
                       fill=orange!30
                      ]
\tikzstyle{decision} = [diamond,
                        minimum width=3cm,
                        minimum height=1cm,
                        text centered,
                        draw=black,
                        fill=green!30
                      ]
\tikzstyle{arrow} = [thick,->,>=stealth]                